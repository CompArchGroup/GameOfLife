\documentclass[12pt]{article}

\usepackage{fullpage}
\usepackage{verbatim}

\author{Eric Dilmore, Andrew Julian, Corrin Thompson, and Ian Brown}
\title{Final Project Update: The Game of Life}
\date{October 6, 2014}

\begin{document}
  \maketitle

  \section{Introduction}
  So far, we have decided on an interface and we have written pseudocode for
  each of the four sections of our project: file IO, array management, program
  logic, and interface output.

  We will address each of these in turn.

  \section{Text Interface (as a group)}
  \label{s:interface}
  The grid will be \(80 \times 50\) characters. Each row will end with a
  newline. Another line will be used to display the HUD, with stats about the
  current game. The stats we plan to track are the number of living cells, the
  number of dead cells, and the current turn. The following several lines is a
  sample output of one single step. (Prototype output made by Andrew.)

  \verbatiminput{interface.txt}

  Each step should take up 52 lines, and the program will assume that the
  terminal is 52 lines high.

  Each step will take approximately one tenth of a second so that the user can
  see the results of each step. Eventually, it may be possible to speed up or
  slow down this rate.

  \section{Pseudocode}
  \subsection{File IO (Corrin)}
  This opens a file for reading, reads the starting position from the file, and
  closes the text file. The values will either be space or @ characters and will
  be parsed into a boolean grid.

  \verbatiminput{readFromFilePseudocode.txt}

  \subsection{Array Management (Ian)}
  There need to be functions that check adjacent cells to tell whether they are
  alive or dead. These functions will do address calculations on the current
  address and put either a one or a zero in \verb-$v0-.

  \verbatiminput{ArrayManagementPseudo.txt}

  \subsection{Game Logic (Eric)}
  For every turn, the game must loop through each of the cells and check in the
  four cardinal directions and the four diagonals to test how many adjacent
  cells are alive and are dead. Then, the current cell must be turned either on
  or off based on the number of adjacent cells touching it.

  This will be a recursively called function, with the recursion in between when
  the calculation is done and the grid itself is updated. The calculated value
  will be stored in an s register, which will be stored on the stack at the
  beginning of every recursive call and restored to its previous value at the
  end of that function.

  As you recurse down, you must also put the return address on the stack in
  order to preserve the call stack. 

  \verbatiminput{logic.txt}

  \subsection{Interface Output (Andrew)}

  The output function will display to the console what is described in Section
  \ref{s:interface}. It will loop over every byte in the grid array and
  translate a boolean dead/alive value to either a space or an @ symbol.

  \verbatiminput{interfacePseudocode.txt}

  \section{Outcomes}
  We have not yet tested our program, as we do not have actual code yet. This
  should be tested as soon as we learn proper function calls.

  \section{Changes From Proposal}
  Nothing has been changed at this stage in the project.

  \section{Next Steps}
  \subsection{File IO (Corrin)}
  Corrin will make the file input loop, looping through each character and
  deciding whether it is on or off. She will also test the loop using MARS.
  \subsection{Array Management (Ian)}
  Ian will translate the pseudocode into MIPS assembly code and test it with
  sample arrays.
  \subsection{Game Logic (Eric)}
  Eric will translate his pseudocode into MIPS assembly code. Due to its
  reliance on array management, he will not test the code directly. However,
  the recursion will be tested for resiliance.
  \subsection{Interface Output (Andrew)}
  Andrew will translate his pseudocode into MIPS assembly code and test it with
  sample arrays.

\end{document}
